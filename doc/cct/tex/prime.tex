%%%%%%%%%%%%%%%%%%%%%%%%%%%%%%%%%%%%%%%%%%%%%%%%%%%%%%%%%%%%%%%%%%%%%%%%%%%

\section{Find largest prime factor}


%%%%%%%%%%%%%%%%%%%%%%%%%%%%%%%%%%%%%%%%%%%%%%%%%%%%%%%%%%%%%%%%%%%%%%%%%%%

\subsection*{Problem}

Determine the largest prime factor of a positive integer.\footnote{
  See Wikipedia for more detail:
  \url{https://en.wikipedia.org/wiki/Integer_factorization}
}


%%%%%%%%%%%%%%%%%%%%%%%%%%%%%%%%%%%%%%%%%%%%%%%%%%%%%%%%%%%%%%%%%%%%%%%%%%%

\subsection*{Solution}

Let $n > 1$ be an integer.  If $n$ is prime, then $n$ is its own
largest prime factor.  Otherwise, use trial division to find a
positive factor of $n$, as shown in \Algorithm{alg:prime:factor}.
Other efficient factorization techniques are available, but trial
division is simple enough and sufficient for the game.  If $n$ is
even, then $2$ is a factor and we are done. Otherwise suppose $n$ is
odd.  The integer $n$ can be factorized as $n = ab$, where $a \geq 1$
and $b \geq 1$.  If $n$ is a perfect square, then $n = ab = a^2$ so
$a = b = \sqrt{n}$.  If $n$ is not a perfect square, then one of the
factors is less than $\sqrt{n}$ and the other is greater than
$\sqrt{n}$.  The idea of trial division is to divide $n$ by odd
integers between $3$ and $\sqrt{n}$, inclusive.  Given an odd integer
$k$ such that $3 \leq k \leq \sqrt{n}$, note the remainder when $n$ is
divided by $k$.  If the remainder is $0$, then $k$ is a factor of $n$
so $n$ is not prime.  Otherwise, set $k \gets k + 2$ and repeat the
division.  If $k > \sqrt{n}$, then $n$ is prime.  Use trial division
to determine all prime factors of $n$ and choose the largest of these
prime factors.

\begin{algorithm}[!htbp]
%%%%%%%%%%%%%%%%%%%%%%%%%%%%%%%%%%%%%%%%%%%%%%%%%%%%%%%%%%%%%%%%%%%%%%%%%%%

\begin{algorithmic}[1]
%%
%% Input.
\Require An integer $n > 1$.
%%
%% Output.
\Ensure A positive factor of $n$.
%%
%% Algorithm body.
%%
%% Check a set of small primes.
\State $\Prime \gets \Set{\TupleD{2}{3}{5}{7}}$\Comment{Check a set of small primes.}
\If{$n \in \Prime$}
  \State \Return $n$
\EndIf
%%
%% n is even
\If{$n$ is even}\Comment{Check whether $n$ is even.}
  \State \Return $2$
\EndIf
%%
%% Trial division.
\State $k \gets 3$\Comment{Trial division.}
\While{$k \leq \sqrt{n}$}
  \If{$n \bmod k = 0$}
    \State \Return $k$\Comment{Found a factor of $n$.}
  \EndIf
  \State $k \gets k + 2$
\EndWhile
\State \Return $n$\Comment{The input $n$ is prime.}
\end{algorithmic}

\caption{%%
  Determine a factor of an integer.
}
\label{alg:prime:factor}
\end{algorithm}

\begin{algorithm}[!htbp]
%%%%%%%%%%%%%%%%%%%%%%%%%%%%%%%%%%%%%%%%%%%%%%%%%%%%%%%%%%%%%%%%%%%%%%%%%%%

\begin{algorithmic}[1]
%%
%% Input.
\Require An integer $n > 1$.
%%
%% Output.
\Ensure The largest prime factor of $n$.
%%
%% Algorithm body.
%%
%% Ensure n is not even.
\State $k \gets n$
\State $p \gets k$
\While{$k \bmod 2 = 0$}\Comment{Ensure $n$ is not even.}
  \State $k \gets k / 2$
  \State $p \gets 2$
\EndWhile
%%
\State $i \gets 3$\Comment{Trial division by odd integers.}
\While{$i \leq \sqrt{n}$}
  \While{$k \bmod i = 0$ \And $k > 1$}
    \State $k \gets k / i$
    \State $p \gets i$
  \EndWhile
  \State $i \gets i + 2$
\EndWhile
\If{$k > 1$}
  \State $p \gets k$
\EndIf
\State \Return $p$
\end{algorithmic}

\caption{%%
  The largest prime factor of an integer.
}
\label{alg:prime:max_prime_factor}
\end{algorithm}

Trial division can be adapted into an algorithm that determines the
largest prime factor of an integer $n > 1$.  The main idea is to
divide $n$ by all primes between $2$ and $\sqrt{n}$, inclusive.  The
game generates a random integer between $500$ and $10^9$, inclusive,
meaning we need to consider all primes at most $31,622$.  We can use
the Sieve of Eratosthenes to generate the next prime as we need it or
use a list of known primes.  A simple solution, as shown in
\Algorithm{alg:prime:max_prime_factor}, is to adapt trial division to
divide $n$ by odd integers between $3$ and $\sqrt{n}$, inclusive.  Set
$k \gets n$.  If $k$ is even, divide $k$ by $2$ and set
$k \gets k / 2$.  If the result is also even, repeat the division by
$2$ until $k$ is no longer even.  So far the highest prime we have
considered is $2$ so we set $p \gets 2$.  The next higher prime is
$i \gets 3$.  If $k$ is divisible by $i$, then divide and assign
(i.e.~the operation $k \gets k / i$) as many times as necessary until
the resulting value of $k$ is no longer divisible by $i$.
Furthermore, we update the value of $p$ to $i$.  As the newer value of
$k$ is no longer divisible by $i$, we update $i$ to the next odd
integer.  Thus we have divided $n$ by the primes in
$\Set{\TupleB{2}{3}}$ and the current value of $k$ is not divisible by
any of these primes.

Let's generalize the above argument.  Let
$P = \SetFirstPrimes{p}{\ell}$ be a set of the first $\ell$ primes,
where $\Seqlt{p}{\ell}$ for some integer $\ell \geq 1$.  Suppose that
$p_{\ell}$ is the highest prime factor of $n$ so far.  Divide $n$ by
each of these primes as many times as necessary until we arrive at an
integer $k \leq n$ that cannot be divided by any prime in $P$.  Let
$i$ be the smallest odd integer greater than $p_{\ell}$ and suppose
that $i$ is a factor of $k$.  We claim that $i$ is the smallest prime
greater than
$p_{\ell}$.  Assume for contradiction that $i$ is not prime.  Then $i$
is a factor of at least two primes in $P$.  Since $i$ is a factor of
$k$, this implies that $k$ is divisible by a prime in $P$,
contradicting the assumption that $k$ is not divisible by any prime in
$P$.  The above argument proves
\Theorem{thm:prime:next_smallest_prime}.

\begin{theorem}
\label{thm:prime:next_smallest_prime}
Let $P = \SetFirstPrimes{p}{\ell}$ be a set of the first $\ell$
primes, where $\Seqlt{p}{\ell}$ for some integer $\ell \geq 1$.  Let
$k > 1$ be an integer that is not divisible by any prime in $P$.
Suppose $i$ is the smallest odd integer greater than $p_{\ell}$.  If
$i$ is a factor of $k$, then $i$ is the smallest prime greater than
$p_{\ell}$.
\end{theorem}

\Theorem{thm:prime:next_smallest_prime} allows us to divide $n$ by
successively larger primes.  As we do so, we keep track of the largest
prime factor so far.  If $k$ is not divisible by any prime at most
$\sqrt{k}$, then $k$ itself is the largest prime factor of $n$.  The
worst case is where $n > 2$ itself is prime.
\Algorithm{alg:prime:max_prime_factor} would attempt to divide $n$ by
all odd integers between $3$ and $\sqrt{n}$, inclusive, before telling
us that $n$ is prime.