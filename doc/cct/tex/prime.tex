%%%%%%%%%%%%%%%%%%%%%%%%%%%%%%%%%%%%%%%%%%%%%%%%%%%%%%%%%%%%%%%%%%%%%%%%%%%

\section{Find largest prime factor}


%%%%%%%%%%%%%%%%%%%%%%%%%%%%%%%%%%%%%%%%%%%%%%%%%%%%%%%%%%%%%%%%%%%%%%%%%%%

\subsection*{Problem}

Determine the largest prime factor of a positive integer.\footnote{
  See Wikipedia for more detail:
  \url{https://en.wikipedia.org/wiki/Integer_factorization}
}


%%%%%%%%%%%%%%%%%%%%%%%%%%%%%%%%%%%%%%%%%%%%%%%%%%%%%%%%%%%%%%%%%%%%%%%%%%%

\subsection*{Solution}

Let $n > 1$ be an integer.  If $n$ is prime, then $n$ is its own
largest prime factor.  Otherwise, use trial division to find a
positive factor of $n$.  Other efficient factorization techniques are
available, but trial division is simple enough and sufficient for the
game.  If $n$ is even, then $2$ is a factor and we are done.
Otherwise suppose $n$ is odd.  The integer $n$ can be factorized as
$n = ab$, where $a \geq 1$ and $b \geq 1$.  If $n$ is a perfect
square, then $n = ab = a^2$ so one of the factors $a$ and $b$ is at
most $\sqrt{n}$.  The idea of trial division is to divide $n$ by odd
integers between $3$ and $\sqrt{n}$, inclusive.  Given an odd integer
$k$ such that $3 \leq k \leq \sqrt{n}$, note the remainder when $n$ is
divided by $k$.  If the remainder is $0$, then $k$ is a factor of $n$
so $n$ is not prime.  Otherwise, set $k \gets k + 2$ and repeat the
division.  If $k > \sqrt{n}$, then $n$ is prime.  Use trial division
to find all prime factors of $n$ and choose the largest of these prime
factors.

\begin{algorithm}[!htbp]
%%%%%%%%%%%%%%%%%%%%%%%%%%%%%%%%%%%%%%%%%%%%%%%%%%%%%%%%%%%%%%%%%%%%%%%%%%%

\begin{algorithmic}[1]
%%
%% Input.
\Require An integer $n > 1$.
%%
%% Output.
\Ensure A positive factor of $n$.
%%
%% Algorithm body.
%%
%% Check a set of small primes.
\State $\Prime \gets \Set{\TupleD{2}{3}{5}{7}}$\Comment{Check a set of small primes.}
\If{$n \in \Prime$}
  \State \Return $n$
\EndIf
%%
%% n is even
\If{$n$ is even}\Comment{Check whether $n$ is even.}
  \State \Return $2$
\EndIf
%%
%% Trial division.
\State $k \gets 3$\Comment{Trial division.}
\While{$k \leq \sqrt{n}$}
  \If{$n \bmod k = 0$}
    \State \Return $k$\Comment{Found a factor of $n$.}
  \EndIf
  \State $k \gets k + 2$
\EndWhile
\State \Return $n$\Comment{The input $n$ is prime.}
\end{algorithmic}

\caption{%%
  Determine a factor of an integer.
}
\label{alg:prime:is_prime}
\end{algorithm}
