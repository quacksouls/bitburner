%%%%%%%%%%%%%%%%%%%%%%%%%%%%%%%%%%%%%%%%%%%%%%%%%%%%%%%%%%%%%%%%%%%%%%%%%%%

\section{Total ways to sum II}


%%%%%%%%%%%%%%%%%%%%%%%%%%%%%%%%%%%%%%%%%%%%%%%%%%%%%%%%%%%%%%%%%%%%%%%%%%%

\subsection*{Problem}

Let $n > 0$ is an integer and consider an array $A$ of positive
integers.  Elements in $A$ are unique.  Determine the number of
different ways to represent $n$ as a sum of integers from $A$.  Each
integer in $A$ can be used zero or more times.


%%%%%%%%%%%%%%%%%%%%%%%%%%%%%%%%%%%%%%%%%%%%%%%%%%%%%%%%%%%%%%%%%%%%%%%%%%%

\subsection*{Solution}

This is a special case of the problem
in \Section{sec:sum:total_ways_to_sum}, known as the coin changing
problem.  Suppose we have a target amount of money $n$.  We have $m$
distinct coins.  The denominations are given by the set
$A = \Set{\Seq{a}{m}}$, where $a_i < a_j$ whenever $i < j$.  How many
ways are there to combine one or more coins such that the total amount
is $n$?  We can use a coin zero, one, or more times.\footnote{
  Refer to this page for more detail:
  \url{https://algorithmist.com/wiki/Coin_change}
}
The problem is easily solved by the method of generating
function.\footnote{
  \url{https://en.wikipedia.org/wiki/Generating_function}
}
The generating function for the coin changing problem is
\[
C(z)
=
\prod_{i=1}^{m}
\frac{
  1
}{
  1 - z^{a_i}
}.
\]
The number of ways to combine coins to reach the target value $n$ is
the coefficient of $z^n$ in the generating function $C(z)$.  How would
we determine the coefficient of $z^n$?  Define $k$ as the least common
multiple of $\Seq{a}{m}$ and define the number
\[
M
=
(k - a_1)
+
(k - a_2)
+
\cdots
+
(k - a_m).
\]
Then the coefficient of $z^n$ is the given by the summation\footnote{
  Refer to the following paper for the derivation:
  \url{https://arxiv.org/abs/1406.5213}
}
\[
\sum_{\substack{0 \; \leq \; i \; \leq \; M\\[2pt] \Equivmod{i}{n}{k}}}
a_i
\binom{
  \frac{n-i}{k} + m - 1
}{
  m - 1
}.
\]

Another way to calculate the coefficient of $z^n$ is to consider each
denomination in turn.  Let $c_i$ be the number of ways to sum to $i$
by using the denominations in $A$.  Define the recurrence relation
%%
\begin{equation}
\label{eqn:sum2:recurrence_relation}
c_i
\coloneqq
c_i + c_{i-a}
\end{equation}
%%
where $a \in A$ and $a \leq i \leq n$.  The initial conditions are
that $c_0 = 1$ and $c_i = 0$ for all $i > 0$.  As we work through each
denomination in turn, the recurrence
\Relation{eqn:sum2:recurrence_relation} gradually propagates values up
to $c_n$.  Once we have worked through all denomination values, the
final value of $c_n$ is the coefficient of $z^n$.  The above procedure
is summarized in \Algorithm{alg:sum2:recurrence_relation}.  Notice the
similarity between
\AlgorithmB{alg:sum:recurrence_relation}{alg:sum2:recurrence_relation}.

\begin{algorithm}[!htbp]
%%%%%%%%%%%%%%%%%%%%%%%%%%%%%%%%%%%%%%%%%%%%%%%%%%%%%%%%%%%%%%%%%%%%%%%%%%%

\section{Total ways to sum II}


%%%%%%%%%%%%%%%%%%%%%%%%%%%%%%%%%%%%%%%%%%%%%%%%%%%%%%%%%%%%%%%%%%%%%%%%%%%

\subsection*{Problem}

Let $n > 0$ is an integer and consider an array $A$ of positive
integers.  Elements in $A$ are unique.  Determine the number of
different ways to represent $n$ as a sum of integers from $A$.  Each
integer in $A$ can be used zero or more times.

\caption{%%
  The number of ways to change money.
}
\label{alg:sum2:recurrence_relation}
\end{algorithm}
