%%%%%%%%%%%%%%%%%%%%%%%%%%%%%%%%%%%%%%%%%%%%%%%%%%%%%%%%%%%%%%%%%%%%%%%%%%%

\section{Total ways to sum II}


%%%%%%%%%%%%%%%%%%%%%%%%%%%%%%%%%%%%%%%%%%%%%%%%%%%%%%%%%%%%%%%%%%%%%%%%%%%

\subsection*{Problem}

Let $n > 0$ is an integer and consider an array $A$ of positive
integers.  Elements in $A$ are unique.  Determine the number of
different ways to represent $n$ as a sum of integers from $A$.  Each
integer in $A$ can be used zero or more times.


%%%%%%%%%%%%%%%%%%%%%%%%%%%%%%%%%%%%%%%%%%%%%%%%%%%%%%%%%%%%%%%%%%%%%%%%%%%

\subsection*{Solution}

This is also known as the coin changing problem.  Suppose we have a
target amount of money $n$.  We have $m$ distinct coins.  The
denominations are given by the set $A = \Set{\Seq{a}{m}}$, where
$a_i < a_j$ whenever $i < j$.  How many ways are there to combine one
or more coins such that the total amount is $n$?  We can use a coin
zero, one, or more times.\footnote{
  Refer to this page for more detail:
  \url{https://algorithmist.com/wiki/Coin_change}
}
The problem is easily solved by the method of generating
function.\footnote{
  \url{https://en.wikipedia.org/wiki/Generating_function}
}
The generating function for the coin changing problem is
\[
C(z)
=
\prod_{i=1}^{m}
\frac{
  1
}{
  1 - z^{a_i}
}.
\]
The number of ways to combine coins to reach the target value $n$ is
the coefficient of $z^n$ in the generating function $C(z)$.  How would
we determine the coefficient of $z^n$?  Define $k$ as the least common
multiple of $\Seq{a}{m}$ and define the number
\[
M
=
(k - a_1)
+
(k - a_2)
+
\cdots
+
(k - a_m).
\]
Then the coefficient of $z^n$ is the given by the summation\footnote{
  Refer to the following paper for the derivation:
  \url{https://arxiv.org/abs/1406.5213}
}
\[
\sum_{\substack{0 \; \leq \; i \; \leq \; M\\[2pt] \Equivmod{i}{n}{k}}}
a_i
\binom{
  \frac{n-i}{k} + m - 1
}{
  m - 1
}.
\]
% Let n be our target sum and let our m denominations be
% D := [d_0, d_1, ..., d_{m-1}]
% where d_i < d_j whenever i < j. That is, the array of denominations is sorted in ascending order.
% We make the following assumptions:
